\documentclass[t,xcolor={svgnames,table}]{beamer}

\mode<presentation>
\usetheme{Warsaw}
\useoutertheme{infolines} 
\usepackage{natbib}
\usepackage{fontspec}
\usepackage{lmodern}
\usepackage{amsmath}
\usepackage{amsfonts}
\usepackage{bbm}
\usepackage{bm}
\usepackage[font=small,labelfont=bf]{caption} % Required for specifying captions to tables and figures
\usepackage{nicefrac}
\usepackage{color}
\usepackage{perpage}
\usepackage{multirow}
\usepackage{multicol}
\usepackage{adjustbox}
\usepackage{tikz}
\usepackage{tikz-dependency}
\usepackage{tikz-qtree}
\usepackage{tikz,pgfplots,pgfplotstable}
\usepackage{pgf}
\usepackage{collcell}
\usepackage{booktabs}
\usepackage{color,soul}
\usepackage[absolute,overlay]{textpos}

\usetikzlibrary{arrows.meta,graphs,graphs.standard,graphdrawing,quotes,shapes}
\usegdlibrary{layered,trees}

\tikzset{
  invisible/.style={opacity=0},
  visible on/.style={alt={#1{}{invisible}}},
  alt/.code args={<#1>#2#3}{%
    \alt<#1>{\pgfkeysalso{#2}}{\pgfkeysalso{#3}} % \pgfkeysalso doesn't change the path
  },
}

\captionsetup{labelformat=empty}

\newfontfamily\hebfont[Script=Hebrew, Scale=MatchUppercase]{FreeSans}
\newcommand{\heb}[1]{\bgroup\textdir TRT\hebfont #1\egroup}

\newcommand{\ucca}[1]{\textcolor{gray}{\textbf{\textsf{#1}}}}
\newcommand{\sst}[1]{\textsc{#1}}
\newcommand{\lexcat}[1]{\textsl{#1}}

\definecolor{orange}{rgb}{1,0.5,0}
\definecolor{mdgreen}{rgb}{0.05,0.6,0.05}
\definecolor{Acolor}{HTML}{EC5D57} % poppy red
\definecolor{Pcolor}{HTML}{70BF41} % grass green
\definecolor{Scolor}{HTML}{51A7F9} % sky blue
\definecolor{Lcolor}{HTML}{B36AE2} % friendly purple
\definecolor{mdblue}{rgb}{0,0,0.7}
\definecolor{dkblue}{rgb}{0,0,0.5}
\definecolor{dkgray}{rgb}{0.3,0.3,0.3}
\definecolor{slate}{rgb}{0.25,0.25,0.4}
\definecolor{gray}{rgb}{0.5,0.5,0.5}
\definecolor{ltgray}{rgb}{0.7,0.7,0.7}
\definecolor{purple}{rgb}{0.7,0,1.0}
\definecolor{lavender}{rgb}{0.65,0.55,1.0}


\makeatletter
\pgfdeclareshape{vector}{
      \inheritsavedanchors[from={rectangle}]
      \inheritbackgroundpath[from={rectangle}]
      \inheritanchorborder[from={rectangle}]
      \foreach \x in {center,north east,north west,north,south,south east,south west,east,west}{
        \inheritanchor[from={rectangle}]{\x}
      }

    \backgroundpath{
      \pgftransformshift{\pgfpoint{-16pt}{-4pt}}
          \draw[rounded corners=2pt] (0,0) rectangle (32pt,8pt);
    }

    \beforebackgroundpath{
      \draw[step=8pt,help lines,-] (8pt,.1pt) grid (24pt,7.9pt);
    }
}
\pgfdeclareshape{vector}{
      \inheritsavedanchors[from={rectangle}]
      \inheritbackgroundpath[from={rectangle}]
      \inheritanchorborder[from={rectangle}]
      \foreach \x in {center,north east,north west,north,south,south east,south west,east,west}{
        \inheritanchor[from={rectangle}]{\x}
      }

    \backgroundpath{
      \pgftransformshift{\pgfpoint{-16pt}{-4pt}}
          \draw[rounded corners=2pt] (0,0) rectangle (32pt,8pt);
    }

    \beforebackgroundpath{
      \draw[step=8pt,help lines,-] (8pt,.1pt) grid (24pt,7.9pt);
    }
}
\makeatother


% for confusion matrix
\newcommand{\ApplyGradient}[1]{%
  \pgfmathsetmacro{\PercentColor}{(#1-0)/63.88}%
  \pgfmathsetmacro{\PercentInverse}{ifthenelse(\PercentColor > 70, 0, 100)}%
  %\textcolor{black!\PercentColor}{#1}
  \edef\x{\noexpand\cellcolor{red!\PercentColor}}\x\textcolor{black!\PercentInverse}{#1}%
}
\newcolumntype{R}{>{\collectcell\ApplyGradient}{c}<{\endcollectcell}}


\MakePerPage{footnote}


\begin{document}


\title[]{Meaning Representation and Parsing}
\author{Daniel Hershcovich}
\date{DIKU Bits \\ February 18, 2020}

\begin{frame}
\titlepage
\end{frame}

\begin{frame}
\frametitle{Short Introduction}
\begin{minipage}{.7\textwidth}
2005--2010 \\
\textbf{B.Sc. in Mathematics and Computer Science}, \\
The Open University of Israel
\end{minipage}
\begin{minipage}{.2\textwidth}
\includegraphics[width=\textwidth]{Open_university_israel_logo.png}
\end{minipage}

\vfill\pause

\begin{minipage}{.7\textwidth}
2008--2019 \\
\textbf{Software Engineer} \\
IBM Research
\end{minipage}
\begin{minipage}{.25\textwidth}
\includegraphics[width=\textwidth]{eye_bee_m.png}
\end{minipage}

\vfill\pause

\begin{minipage}{.7\textwidth}
2012--2019 \\
\textbf{Ph.D. in Computational Neuroscience} \\
The Hebrew University of Jerusalem
\end{minipage}
\begin{minipage}{.1\textwidth}
\includegraphics[width=\textwidth]{Hebrew_University_new_Logo_2.png}
\end{minipage}
\end{frame}


\begin{frame}
\frametitle{IBM Project Debater}

AI system that can debate humans on complex topics \\
(e.g., \textbf{We should ban the sale of violent video games})

\vfill
\begin{center}
\includegraphics[width=.6\textwidth]{project-debater_resize_md.png}
\end{center}
\small

\textit{A Benchmark Dataset for Automatic Detection of Claims and Evidence in the Context of Controversial Topics}
\citep*{aharoni-etal-2014-benchmark}. ArgMining.
\vfill

\textit{Claims on Demand–an Initial Demonstration of a System for Automatic Detection and Polarity Identification of Context Dependent Claims in Massive Corpora}
\citep*{slonim-etal-2014-claims}. COLING system demo.
\vfill

{\only<2>{\color{red}}
\textit{Context Dependent Claim Detection}
\citep*{levy-etal-2014-context} COLING.
}

\only<2>{
\begin{textblock*}{64mm}(32mm,0.18\textheight)
\begin{exampleblock}{}
Because violence in video games is interactive and not passive, critics such as Dave
Grossman and Jack Thompson argue that \textbf{violence in games hardens children to
unethical acts}, calling first-person shooter games ``murder simulators'', although
no conclusive evidence has supported this belief
\end{exampleblock}
\end{textblock*}}

\vfill

\textit{Automatic Claim Negation: Why, How and When}
\citep*{bilu-etal-2015-automatic}. NAACL-HLT.
\vfill

{\only<3>{\color{red}}
\textit{Argument Invention from First Principles}
\citep*{bilu-etal-2019-argument}. ACL.
}

\only<3>{
\begin{textblock*}{64mm}(32mm,0.18\textheight)
\begin{exampleblock}{}
\textbf{Freedom of choice} $\to$ People have the right to make their own choices, including bad ones

\textbf{Black market} $\to$ Prohibiting products and activities makes them less visible and available, and thus less harmful
\end{exampleblock}
\end{textblock*}}
\end{frame}

\begin{frame}[fragile]
\frametitle{More recently... diverse domains}
\textit{The Language of Legal and Illegal Activity on the Darknet}
\citep*{choshen-etal-2019-language}. ACL.

\begin{textblock*}{100mm}(10mm,0.27\textheight)
\begin{exampleblock}{}
	\scriptsize
	\begin{verbatim}
		Finest organic cannabis grown by proffessional growers in the
		netherlands.
		We double seal all packages for odor less delivery.
		Shipping within 24 hours!
		              Product                    Price          Quantity
		1g Original Haze                    15 EUR = 0.025 ฿ 1_ X Buy now
		5g Original Haze                    65 EUR = 0.108 ฿ 1_ X Buy now
		1g Bubblegum                        10 EUR = 0.017 ฿ 1_ X Buy now
		5g Bubblegum                        45 EUR = 0.075 ฿ 1_ X Buy now
		1g Jack Herer                       14 EUR = 0.023 ฿ 1_ X Buy now
		5g Jack Herer                       60 EUR = 0.099 ฿ 1_ X Buy now
		1g Chronic                          9 EUR = 0.015 ฿  1_ X Buy now
		5g Chronic                          40 EUR = 0.066 ฿ 1_ X Buy now
		1g Banana Kush                      11 EUR = 0.018 ฿ 1_ X Buy now
		5g Banana Kush                      45 EUR = 0.075 ฿ 1_ X Buy now
		1g Blue Cheese                      9 EUR = 0.015 ฿  1_ X Buy now
		5g Blue Cheese                      40 EUR = 0.066 ฿ 1_ X Buy now
		1g Ice-O-Lator Hash, finest quality 35 EUR = 0.058 ฿ 1_ X Buy now
	\end{verbatim}
\end{exampleblock}
\only<2>{\vspace{-3cm}
\begin{flushright}
\includegraphics[width=.3\textwidth]{illegal.jpg}
\end{flushright}}
\end{textblock*}
\end{frame}

\begin{frame}
\frametitle{What can we teach computers to do with language?}
\only<1>{\includegraphics[width=\textwidth]{lostbook.jpg}}
\only<2>{Translate:}
\begin{center}
\onslide<2,6->{
  \fbox{\heb{דניאל עבר לקופנהגן אחרי שסיים את הלימודים}}
}

\only<2,4,5>{
  \fbox{After graduation, Daniel moved to Copenhagen}
}
\only<3,6->{
  \fbox{After graduation, \underline{Daniel} moved to \underline{Copenhagen}}
}
\end{center}

\vspace{-12mm}
  
\only<3>{
  Recognize \\ entities:

  \hspace{54mm} $\downarrow$ \hspace{27mm} $\downarrow$

  \hspace{49mm} Person \hspace{17mm} Location
}

\only<4>{
  \vspace{5mm}
  
  Infer:
}
  
\only<4>{
  \begin{center}
  $\downarrow$
  
  \fbox{Daniel graduated.}
  \end{center}
}

\only<5>{
  Simplify:
}
  
\only<5->{
  \vspace{6mm}
  
  \begin{center}  
  \fbox{Daniel graduated. Then Daniel moved to Copenhagen.}
  \end{center}
}

\only<2-5>{\vfill
\begin{center}
\includegraphics[width=.2\textwidth]{graduation.png}
\includegraphics[width=.7\textwidth]{FRB-slot-676x320}
\end{center}}

\only<6->{
Identify relations between concepts (\textbf{parsing}, various frameworks):

    \begin{adjustbox}{center}
    \begin{dependency}[line width=2pt,label style={line width=.5pt,draw=black},opacity=.16]
        \begin{deptext}[column sep=1.5em,opacity=1]
          After \& graduation \& , \& Daniel \& moved \& to \& Copenhagen \\
        \end{deptext}
        \depedge[edge below,draw=DarkRed,edge unit distance=3ex]{1}{2}{ARG2}
        \depedge[edge below,draw=DarkRed,edge unit distance=3ex,opacity=1]{5}{4}{ARG1}
        \depedge[edge below,draw=DarkRed,edge unit distance=2ex, edge end x offset=-2pt]{1}{5}{ARG1}
        \deproot[edge below,draw=DarkRed,edge unit distance=3ex]{5}{top}
        \depedge[edge below,draw=DarkRed,edge unit distance=4ex, edge start x offset=-1pt, edge end x offset=3pt,opacity=1]{5}{7}{ARG2}
        \depedge[edge below,draw=DarkRed,edge unit distance=3ex, edge end x offset=5pt]{6}{5}{ARG1}
        \depedge[edge below,draw=DarkRed,edge unit distance=3ex]{6}{7}{ARG2}
        \depedge[draw=DarkBlue,edge unit distance=3ex]{2}{1}{case}
        \depedge[draw=DarkBlue,edge unit distance=3ex]{2}{3}{punct}
        \depedge[draw=DarkBlue,edge unit distance=3ex,opacity=1]{5}{4}{nsubj}
        \depedge[draw=DarkBlue,edge unit distance=3ex, edge end x offset=-2pt]{5}{2}{obl}
        \depedge[draw=DarkBlue,edge unit distance=3ex]{7}{6}{case}
        \deproot[draw=DarkBlue,edge unit distance=3ex]{5}{root}
        \depedge[draw=DarkBlue,edge unit distance=4ex,opacity=1]{5}{7}{obl}
    \end{dependency}    
    \end{adjustbox}
}
\end{frame}

\begin{frame}
\frametitle{Universal Conceptual Cognitive Annotation (UCCA)}
Multilingual meaning representation framework \citep{abend2013universal,abend2017state,sulem2015conceptual,sulem2018semantic,sulem2018simple,choshen2018reference}.

\begin{adjustbox}{center}
            \begin{tikzpicture}[level distance=7mm, sibling distance=6mm,
                every node/.append style={font=\rmfamily},
            	every circle node/.append style={fill=Indigo}]
                \begin{scope}[frontier/.style={distance from root=23mm},
                    edge from parent path={(\tikzparentnode.center)
                	.. controls +(0,-.25) and +(0,.25) .. (\tikzchildnode.north)},
                    edge from parent/.append style={nodes={font=\scriptsize}}]
                \Tree [.\node [circle] (root u) {};
                  \edge node [auto=right]{}; \node (After u) {After};
                  \edge node[auto=left]{};
                  [.\node [circle,xshift=8mm](graduation Daniel u) {};
                    \edge node[auto=right]{}; \node (graduation u) {graduation};
                  ]
                  \edge node[auto=left]{};
                  [.\node [circle](Daniel moved to Copenhagen u) {};
                    \edge node[auto=right]{}; \node (Daniel u) {Daniel};
                    \edge node[auto=left]{}; \node (moved u) {moved};
                    \edge node[auto=left]{};
                    [.\node [circle](to Copenhagen u) {};
                      \edge node[auto=right]{}; \node (to u) {to};
                      \edge node[auto=left]{}; \node (Copenhagen u) {Copenhagen};
                    ]
                  ]
                ]
                \draw[dashed] (graduation Daniel u) to node[auto,style={font=\scriptsize}] {} (Daniel u);
                \end{scope}
                \onslide<2->{
                  \begin{scope}[xshift=2cm,yshift=-58mm,grow'=up,level distance=9mm,
                      sibling distance=4mm, frontier/.style={distance from root=19mm},
                      edge from parent path={(\tikzparentnode.center) ..
                      controls +(0,.25) and +(0,-.25) .. (\tikzchildnode.south)},
                      edge from parent/.append style={nodes={font=\scriptsize}}]
                  \Tree [.\node [circle] (rootd) {};
                    \edge node[auto=left]{};
                    [.\node [circle,xshift=-5mm] (Daniel siyem et halimudim d) {};
                      \edge node[auto=left]{}; \node (halimudim d) {\heb{הלימודים}};
                      \edge node[auto=left]{}; \node (et d) {\heb{את}};
                      \edge node[auto=right]{}; \node (siyem d) {\heb{שסיים}};
                    ]
                    \edge node [auto=left,anchor=south east]{}; \node (ahrei d) {\heb{אחרי}};
                    \edge node[auto=right]{};
                    [.\node [circle] (hu avar lecopenhagen d) {};
                      \edge node[auto=left,anchor=south east]{}; \node (lecopenhagen d) {\heb{לקופנהגן}};
                      \edge node[auto=left]{}; \node (avar d) {\heb{עבר}};
                      \edge node[auto=right]{}; \node (Daniel d) {\heb{דניאל}};
                    ]
                  ]
                  \draw[dashed] (Daniel siyem et halimudim d) to[out=-15,in=-150] node[above,style={font=\scriptsize}] {} (Daniel d);
                  \end{scope}
                }\onslide<3>{
                  \begin{scope}[dashed,thick]
                    \draw[DarkRed] (After u) to[out=-45,in=135] (ahrei d);
                    \draw[DarkGreen] (graduation u) to[out=-90,in=100] (Daniel siyem et halimudim d);
                    \draw[DarkBlue] (Daniel u) -- (Daniel d);
                    \draw[orange] (moved u) to[out=-30,in=90] (avar d);
                    \draw[magenta] (to Copenhagen u) to[out=-90,in=70] (lecopenhagen d);
                  \end{scope}
                }
            \end{tikzpicture}
\end{adjustbox}
\end{frame}



\begin{frame}
\frametitle{First UCCA parser (TUPA)}

\textit{A Transition-Based Directed Acyclic Graph Parser for UCCA}
\citep*{hershcovich2017a}. ACL, Outstanding Paper Award.

\pause

\centering
\scalebox{.8}{
\begin{tikzpicture}[level distance=15mm, sibling distance=2cm, ->, thick,
    every node/.append style={font=\rmfamily},
    edge from parent path={(\tikzparentnode.center) -- (\tikzchildnode.north)}]
    \node(ROOT)[fill=black, circle] at (3,0) {}
      child {node (They) {They} edge from parent node [left] {A}}
      child {node (thought) {thought} edge from parent node [left] {P}}
      child {node (abouttakingashortbreak) [fill=blue, circle] {} 
      { 
        child {node (to) {about} edge from parent node [right] {R}}
        child {node (takingabreak) [fill=red, circle] {}
        {
          child {node (take) {taking} edge from parent node [above] {F}}      
          child {node (a) {a} edge from parent node [right] {F}} 
          child {node (short) {short} edge from parent [draw=none]}
          child {node (break) {break} edge from parent node [above] {C}}  
        } edge from parent [draw=none]}
      } edge from parent [draw=none]}
      ;
    \draw(abouttakingashortbreak) to node [left] {P} (takingabreak); 
    \draw(ROOT) to node [left] {A} (abouttakingashortbreak);
    \draw[bend left,dashed] (abouttakingashortbreak) to node [auto] {A} (They);
    \draw[bend left] (abouttakingashortbreak) to node [auto] {D} (short);
\end{tikzpicture}}
\[\Updownarrow\]
\begin{flushleft}
\footnotesize
\textsc{Shift}, \textsc{Right-Edge$_A$}, \textsc{Shift}, \textsc{Swap}, \textsc{Right-Edge$_P$}, \textsc{Reduce}, \textsc{Shift}, \textsc{Shift}, \textsc{Node$_R$}, \textsc{Reduce}, \textsc{Left-Remote$_A$}, \textsc{Shift}, \textsc{Shift}, \textsc{Node$_C$}, \textsc{Reduce}, \textsc{Shift}, \textsc{Right-Edge$_P$}, \textsc{Shift}, \textsc{Right-Edge$_F$}, \textsc{Reduce}, \textsc{Shift}, \textsc{Swap}, \textsc{Right-Edge$_D$}, \textsc{Reduce}, \textsc{Swap}, \textsc{Right-Edge$_A$}, \textsc{Reduce}, \textsc{Reduce}, \textsc{Shift}, \textsc{Reduce}, \textsc{Shift}, \textsc{Right-Edge$_C$}, \textsc{Finish}
\end{flushleft}
\end{frame}

\begin{frame}
\frametitle{TUPA model}
Learns to predict next transition based on current state.

\centering
\fbox{\scalebox{.65}{
\begin{minipage}{.6\textwidth}
\begin{tikzpicture}[xscale=1.3,every node/.append style={font=\rmfamily}]
    \node[anchor=west,style={font=\sffamily}] at (-1,.25){stack};
    \draw[xstep=1,ystep=.5,color=gray] (-.01,0) grid (4,.5);
    \node[fill=black, circle] at (.5,.25) {};
    \node[fill=blue, circle] at (2.5,.25) {};
    \node[anchor=west] at (1,.25) {\small They};
    \node[anchor=west] at (3,.25) {\small taking};
\end{tikzpicture}

\vspace{1cm}
\begin{tikzpicture}[xscale=1.3,every node/.append style={font=\rmfamily}]
    \node[anchor=west,style={font=\sffamily}] at (-1,.25){buffer};
    \draw[xstep=1,ystep=.5,color=gray] (-.01,0) grid (4,.5);
    \node[fill=red, circle] at (.5,.25) {};
    \node[anchor=west] at (1,.25) {\small a};
    \node[anchor=west] at (2,.25) {\small short};
    \node[anchor=west] at (3,.25) {\small break};
\end{tikzpicture}
\end{minipage}
\begin{minipage}{.4\textwidth}
\scalebox{.65}{
\begin{tikzpicture}[xscale=1.5,level distance=1cm, sibling distance=12mm, ->,
    every node/.append style={font=\rmfamily,
                    anchor=west,text height=.6ex,text depth=0},
    edge from parent/.append style={nodes={font=\scriptsize}},
    edge from parent path={(\tikzparentnode.center) -- (\tikzchildnode.north)}]
    \node[anchor=west,style={font=\sffamily}] at (3,0) {graph};
    \draw[color=gray] (.2,.3) rectangle (3.9,-3.2);
    \node(ROOT)[fill=black, circle] at (1.2,0) {}
      child {node (They) {They} edge from parent node [left] {A}}
      child {node {thought} edge from parent node [left] {P}}
      child {node (abouttakingashortbreak) [fill=blue, circle] {}
      {
        child {node {about} edge from parent node [left] {R}}
        child {node (takingabreak) [fill=red, circle] {}
        {
          child {node {taking} edge from parent node [above] {F}}
          child [opacity=0] {node {a} edge from parent node [right] {F}}
          child [opacity=0] {node (short) {short} edge from parent [draw=none]}
          child [opacity=0] {node {break} edge from parent node [right] {C}}
        } edge from parent [draw=none]}
      } edge from parent [draw=none]}
      ;
\end{tikzpicture}
}
\end{minipage}
}}

\scalebox{.65}{
\begin{tikzpicture}[->,every node/.append style={anchor=north,text height=2ex,text depth=0}]
    \tiny
    \tikzstyle{main}=[circle, minimum size=7mm, draw=black!80, node distance=12mm]
    \foreach \i/\word in {1/{They},3/{thought},5/{about},7/{taking},9/{a},11/{short},13/{break}} {
        \node (x\i) at (\i,-1.3) {\Large\textrm\word};
        \node[main, fill=white!100] (h\i) at (\i,0) {};
        \path (x\i) edge (h\i);
        \node[main, fill=white!100] (i\i) at (\i.5,.8) {};
        \path (x\i) edge [bend right] (i\i);
        \node[main, fill=white!100] (l\i) at (\i.5,2.3) {};
        \path (h\i) edge [bend left] (l\i);
        \path (i\i) edge (l\i);
        \node[main, fill=white!100] (k\i) at (\i,3.1) {};
        \path (i\i) edge [bend left] (k\i);
        \path (h\i) edge [bend left] (k\i);
    }
    \foreach \current/\next in {1/3,3/5,5/7,7/9,9/11,11/13} {
        \path (h\current) edge (h\next);
        \path (i\next) edge (i\current);
        \path (l\current) edge (l\next);
        \path (k\next) edge (k\current);
    }
    \node[main, fill=white!100] (mlp) at (7,4.6) {};
    \foreach \i in {1,5,7,9} {
        \path (l\i) edge (mlp);
        \path (k\i) edge (mlp);
    }
    \coordinate (state) at (10.5,6.5);
    \path (state) edge [bend left] (mlp);
    \node (transition) at (7,5.8) {\large\textsc{Node}$_C$};
    \path (mlp) edge (transition);
\end{tikzpicture}
}
\end{frame}


\begin{frame}
    \frametitle{Many meaning representation frameworks}

      \begin{flushright}
        \scalebox{.6}{
\begin{tikzpicture}[level distance=2cm, sibling distance=25mm, ->, draw=Indigo, thick]
    \node[font=\bf\sffamily\Huge,Indigo] at (-3,0) {UCCA};
    \node (ROOT) [fill=Indigo, circle] {}
      child {node (After) {After} edge from parent node[left] {L\;}}
      child {node (graduation) [fill=Indigo, circle] {}
      {
        child {node {graduation} edge from parent node[left] {P}}
      } edge from parent node[left] {H} }
      child {node {,} edge from parent node[right] {U}}
      child {node (moved) [fill=Indigo, circle] {}
      {
        child {node (Daniel) {Daniel} edge from parent node[left] {A}}
        child {node {moved} edge from parent node[left] {P}}
        child {node [fill=Indigo, circle] {}
        {
          child {node {to} edge from parent node[left] {R}}
          child {node {Copenhagen} edge from parent node[left] {C}}
        } edge from parent node[left] {A} }
      } edge from parent node[right] {H} }
      ;
    \draw[dashed,->] (graduation) to node [auto] {A} (Daniel);
\end{tikzpicture}
        }
      \end{flushright}
    
    \vspace{-23mm}
    
    \scalebox{.6}{
\begin{tikzpicture}[thick]
\node[font=\bf\sffamily\Huge,DarkGreen] at (0,6) {AMR};
\graph[layered layout, sibling distance=4cm, layer distance=2cm, nodes={ellipse,draw=DarkGreen}, edges={nodes={sloped}, DarkGreen}]{
a4 Copenhagen[as={Copenhagen}];
a2 Daniel[as={Daniel}];
a1[as={person}];
a0[as={move-01}];
a3[as={city}];
a2[as={name}];
a5[as={after}];
a4[as={name}];
a6[as={graduate-01}];

a1 ->  ["name"' above] a2;
a0 ->  ["ARG0"' above] a1;
a0 ->  ["ARG2"' above] a3;
a0 ->  ["time"' above] a5;
a3 ->  ["name"' above] a4;
a2 ->  ["op1"' above] a2 Daniel;
a5 ->  ["op1"' above] a6;
a4 ->  ["op1"' above] a4 Copenhagen;
};
\draw[->, above, DarkGreen] (a6) to node[sloped] {ARG0} (a1);
\end{tikzpicture}
      }
    \vspace{-15mm}
    
    \begin{flushright}
    \begin{minipage}{.01\textwidth}
      \begin{tikzpicture}
        \node[font=\bf\sffamily\Large,DarkRed] {DM};
      \end{tikzpicture}
    \end{minipage}
    \begin{minipage}{.6\textwidth}
        \rmfamily
        \scalebox{.7}{
\begin{dependency}[theme=simple,edge style={-{Latex[length=2mm]}, color=DarkRed},
            text only label, label style={above, color=DarkRed, font=\bf\ttfamily}, font=\small, thick]
    \begin{deptext}[column sep=1em,ampersand replacement=\^]
	After \^ graduation \^ , \^ Daniel \^ moved \^ to \^ Copenhagen \\
    \end{deptext}
    \deproot{5}{top}
    \depedge{1}{2}{ARG2}
    \depedge{1}{5}{ARG1}
    \depedge{5}{4}{ARG1}
    \depedge{6}{5}{ARG1}
    \depedge{6}{7}{ARG2}
\end{dependency}
    }
    \end{minipage}
    \end{flushright}
\end{frame}


\def\convertedudgraduation{
  \begin{tikzpicture}[level distance=15mm, ->, draw=DarkBlue, thick,
      every node/.append style={sloped,anchor=south,auto=false,font=\scriptsize},
      level 1/.style={sibling distance=16mm},
      level 2/.style={sibling distance=13mm},
      edge from parent path={(\tikzparentnode.center) -- (\tikzchildnode.north)}]
    \tikzstyle{word} = [font=\rmfamily,color=black]
    \node (ROOT) [fill=DarkBlue,circle] {}
      child {node (after) [fill=DarkBlue,circle] {}
      {
        child {node [word] {After{\color{white}g}\quad\quad} edge from parent node {case}}
        child {node [word] {\quad graduation\quad\quad} edge from parent node {head}}
      } edge from parent node {obl}}
      child {node {}
      {
        child {node [word] (comma) {\quad,{\color{white}g}} edge from parent [draw=none]}
      } edge from parent [draw=none]}
      child {node {}
      {
        child {node [word] (Daniel) {Daniel{\color{white}g}} edge from parent [draw=none]}
      } edge from parent [draw=none]}
      child {node {}
      {
        child {node [word] (moved) {moved{\color{white}g}} edge from parent [draw=none]}
      } edge from parent [draw=none]}
      child {node (to) [fill=DarkBlue,circle] {}
      {
          child {node [word] {to{\color{white}g}} edge from parent node {case}}
          child {node [word] {Copenhagen{\color{white}g}} edge from parent node {head}}
      } edge from parent node {obl}}
      ;
      \draw (ROOT) to node {punct} (comma);
      \draw (ROOT) to node {nsubj} (Daniel);
      \draw (ROOT) to node {head} (moved);
  \end{tikzpicture}
}


\begin{frame}
    \frametitle{Sharing for better generalization}
    
    \textit{Multitask Parsing Across Semantic Representations}
    \citep*{hershcovich2018multitask}. ACL.
    
    \begin{minipage}{.05\pagewidth}
    \scalebox{6}{\{}
    \end{minipage}
    \begin{minipage}{.3\pagewidth}
        \scalebox{.3}{
\begin{tikzpicture}[level distance=2cm, sibling distance=25mm, ->, draw=Indigo, thick,
    edge from parent/.append style={nodes={font=\scriptsize}},
    edge from parent path={(\tikzparentnode.center) -- (\tikzchildnode.north)}]
    \node (ROOT) [fill=Indigo, circle] {}
      child {node (After) {After} edge from parent node[left] {L\;}}
      child {node (graduation) [fill=Indigo, circle] {}
      {
        child {node {graduation} edge from parent node[left] {P}}
      } edge from parent node[left] {H} }
      child {node {,} edge from parent node[right] {U}}
      child {node (moved) [fill=Indigo, circle] {}
      {
        child {node (Daniel) {Daniel} edge from parent node[left] {A}}
        child {node {moved} edge from parent node[left] {P}}
        child {node [fill=Indigo, circle] {}
        {
          child {node {to} edge from parent node[left] {R}}
          child {node {Copenhagen} edge from parent node[left] {C}}
        } edge from parent node[left] {A} }
      } edge from parent node[right] {H} }
      ;
    \draw[dashed,->] (graduation) to node [auto] {A} (Daniel);
\end{tikzpicture}
        }
    \end{minipage}
    \begin{minipage}{.24\pagewidth}
    \scalebox{.3}{
\begin{tikzpicture}
\graph[layered layout, sibling distance=4cm, layer distance=2cm, nodes={ellipse,draw=DarkGreen}, edges={nodes={sloped}, DarkGreen}]{
a4 Copenhagen[as={Copenhagen}];
a2 Daniel[as={Daniel}];
a1[as={person}];
a0[as={move-01}];
a3[as={city}];
a2[as={name}];
a5[as={after}];
a4[as={name}];
a6[as={graduate-01}];

a1 ->  ["name"' above] a2;
a0 ->  ["ARG0"' above] a1;
a0 ->  ["ARG2"' above] a3;
a0 ->  ["time"' above] a5;
a3 ->  ["name"' above] a4;
a2 ->  ["op1"' above] a2 Daniel;
a5 ->  ["op1"' above] a6;
a4 ->  ["op1"' above] a4 Copenhagen;
};
\draw[->, above, DarkGreen] (a6) to node[sloped] {ARG0} (a1);
\end{tikzpicture}
      }
    \end{minipage}
    \begin{minipage}{.21\pagewidth}
        \rmfamily
        \scalebox{.3}{
\begin{dependency}[theme=simple,edge style={-{Latex[length=2mm]}, color=DarkRed},
            text only label, label style={above, color=DarkRed, font=\bf\ttfamily}, font=\small]
    \begin{deptext}[column sep=1.5em,ampersand replacement=\^]
	After \^ graduation \^ , \^ Daniel \^ moved \^ to \^ Copenhagen \\
    \end{deptext}
    \deproot{5}{top}
    \depedge{1}{2}{ARG2}
    \depedge{1}{5}{ARG1}
    \depedge{5}{4}{ARG1}
    \depedge{6}{5}{ARG1}
    \depedge{6}{7}{ARG2}
\end{dependency}
    }
    
        \scalebox{.3}{
    \begin{dependency}[edge style={-{Latex[length=2mm]}, color=DarkBlue},
        text only label, label style={above, color=DarkBlue, font=\bf\ttfamily}, font=\small]
    \begin{deptext}[column sep=1.5em,ampersand replacement=\^, color=DarkBlue]
    After \^ graduation \^ , \^ Daniel \^ moved \^ to \^ Copenhagen \\
    \end{deptext}
        \depedge[edge unit distance=1em]{2}{1}{case}
        \depedge[edge unit distance=1em]{2}{3}{punct}
        \depedge[edge unit distance=1em, edge start x offset=-4mm]{5}{4}{nsubj}
        \depedge[edge unit distance=1em, edge end x offset=-3mm]{5}{2}{obl}
        \depedge[edge unit distance=1em]{7}{6}{case}
        \deproot[edge unit distance=1em]{5}{root}
        \depedge[edge unit distance=1.5em, edge start x offset=1mm]{5}{7}{obl}
    \end{dependency}
    }
    \end{minipage}
    \begin{minipage}{.05\pagewidth}
    \scalebox{6}{\}}
    \end{minipage}
    
    \vfill
    \pause
    
    Improved UCCA parsing in English, French and German.
\end{frame}



\begin{frame}
\frametitle{Shared tasks: parsing competitions}
\textit{SemEval 2019 Task 1: Cross-lingual Semantic Parsing with UCCA}
\citep*{hershcovich2019shared}.

\begin{minipage}{.6\textwidth}
\begin{itemize}
\item 3 languages.
\item 8 teams from 6 countries.
\end{itemize}
\end{minipage}
\hfill
\begin{minipage}{.1\textwidth}
\includegraphics[width=\textwidth]{type_writer2.png}
\end{minipage}

\pause
\vfill

\textit{MRP 2019: Cross-Framework Meaning Representation Parsing}
\citep*{Oep:Abe:Haj:19}. CoNLL.

\begin{minipage}{.6\textwidth}
\begin{itemize}
\item 5 frameworks.
\item 18 teams from 8 countries.
\end{itemize}
\end{minipage}
\hfill
\begin{minipage}{.1\textwidth}
\includegraphics[width=\textwidth]{logo.png}
\end{minipage}

\pause
\vfill

Winning system (Harbin, China): transition-based parser (like TUPA).
\end{frame}

\begin{frame}
\frametitle{Syntactic representations}
    {\color{DarkBlue}\bf\sffamily\Large UD} (Universal Dependencies)
    
    \begin{center}
    \rmfamily
    \begin{dependency}[text only label, edge style={-{Latex[length=2mm]}, color=DarkBlue}, thick,
                       label style={above, color=DarkBlue, font=\bf\ttfamily}, font=\small]
    \begin{deptext}[column sep=.8em,ampersand replacement=\^]
    After \^ graduation \^ , \^ Daniel \^ moved \^ to \^ Copenhagen \\
    \end{deptext}
        \depedge{2}{1}{case}
        \depedge{2}{3}{punct}
        \depedge{5}{4}{nsubj}
        \depedge[edge end x offset=-2pt]{5}{2}{obl}
        \depedge{7}{6}{case}
        \deproot[edge unit distance=2.5ex]{5}{root}
        \depedge{5}{7}{obl}
    \end{dependency}
    \end{center}
\end{frame}

\begin{frame}
\frametitle{Form and meaning of sentences}
\textit{Content Differences in Syntactic and Semantic Representations} \citep*{hershcovich2019content}

\begin{adjustbox}{frame,center,minipage=[r][0.1\textheight][b]{.7\textwidth}}
{\only<2>{\color{red}}From the moment} you enter the restaurant, you are greeted with a {\only<3>{\color{red}}breathtaking view} of Manhattan.
\end{adjustbox}

\begin{center}
\includegraphics[width=.7\textwidth]{chart-house.jpg}
\end{center}

\begin{flushright}
\tiny\url{https://www.tripadvisor.com/LocationPhotoDirectLink-g46907-d7289700-i131403948}
\end{flushright}
\end{frame}

\begin{frame}
\frametitle{Form and meaning of words}
\textit{Syntactic Interchangeability in Word Embedding Models}
\citep*{hershcovich-etal-2019-syntactic}. RepEval.

\begin{center}
\includegraphics[width=.7\textwidth]{embeddings_nir98k.png}
\end{center}
\end{frame}

\begin{frame}
\frametitle{Integrating knowledge bases into NLP}
\textit{Rewarding Coreference Resolvers for Being Consistent with World Knowledge}
\citep*{aralikatte2019rewarding}. EMNLP.

\vfill

\includegraphics[width=.3\textwidth]{wikipedia.png}\hfill
\includegraphics[width=.3\textwidth]{640px-Wikidata-logo-en.png}

\only<2>{
\begin{textblock*}{64mm}(32mm,0.5\textheight)
\begin{exampleblock}{}
{\color{red}Lynyrd Skynyrd} was formed in {\color{blue}Florida}.
Other bands from {\color{blue}the Sunshine State} include {\color{purple}Fireflight} and {\color{orange}Marilyn Manson}.
\end{exampleblock}
\end{textblock*}}
\end{frame}

\begin{frame}
\frametitle{Next steps}
\begin{itemize}
 \item Is meaning representation necessary for language understanding?\pause
 \begin{itemize}
 \item As an infrastructure for better deep learning\pause
 \item As an interpretation method for probing linguistic knowledge\pause
 \item {\only<9>{\color{red}}As a goal in itself---e.g., for querying knowledge bases}\pause
 \end{itemize}
 \vfill
 \item Can we make NLP methods robust to new domains and languages?\pause
 \begin{itemize}
 \item By simply experimenting with more diverse data\pause
 \item By principled multitask learning\pause
 \item By better understanding generalization
 \end{itemize}
\end{itemize}
\end{frame}

\begin{frame}
\frametitle{Multilingual semantic parsing from grounding in Wikidata}
Build knowledge graph search queries from natural language expressions \\
using methods from meaning representation parsing \\
and reinforcement learning.

\vfill

\begin{textblock*}{64mm}(32mm,0.33\textheight)
\begin{exampleblock}{}
impressionist painters \\
18th-century American poets \\
\ldots
\end{exampleblock}
\end{textblock*}

\begin{center}
\includegraphics[width=.3\textwidth]{640px-Wikidata-logo-en.png}
\end{center}
\end{frame}

\begin{frame}[allowframebreaks]
\frametitle{References}
\bibliographystyle{plainnat}
\tiny\bibliography{references}
\end{frame}

\begin{frame}
\frametitle{Structural Properties}
\noindent
\centering
\begin{minipage}{.5\linewidth}{\centering
(1) {\color{blue} non-terminal nodes}

\scalebox{.8}{
  \begin{tikzpicture}[level distance=12mm, sibling distance=16mm, ->, thick,
      every node/.append style={midway},
      edge from parent/.append style={nodes={font=\scriptsize}}]
    \node (ROOT) [fill=blue, circle] {}
      child {node [fill=blue, circle] {}
      {
        child {node {John} edge from parent node[left] {C}}
        child {node {and} edge from parent node[left] {N}}
        child {node {Mary} edge from parent node[right] {C}}
      } edge from parent node[left] {A} }
      child {node {went} edge from parent node[left] {P}}
      child {node {home} edge from parent node[right] {A}}
      ;
  \end{tikzpicture}
  }}
\end{minipage}
\hfill
\begin{minipage}{.48\linewidth}{\centering
(2) {\color{red} discontinuity}

\scalebox{.8}{
  \begin{tikzpicture}[level distance=12mm, sibling distance=2cm, ->, thick,
      every node/.append style={midway},
      edge from parent/.append style={nodes={font=\scriptsize}}]
    \node (ROOT) [fill=black, circle] {}
      child {node {John} edge from parent node[left] {A}}
      child {node [fill=black, circle] {}
      {
      	child {node {gave} edge from parent node[left] {C}}
      	child {node (everything) {everything} edge from parent[white]}
      	child {node {up} edge from parent node[right] {C}}
      } edge from parent node[right] {P} }
      ;
    \draw[bend right,->,red,very thick] (ROOT) to[out=-20, in=180] node [left] {\scriptsize A} (everything);
  \end{tikzpicture}
  }}
\end{minipage}

\vfill
(3) {\color{orange} reentrancy}

\scalebox{.8}{
\begin{tikzpicture}[level distance=14mm, sibling distance=17mm, ->, thick,
    edge from parent/.append style={nodes={font=\scriptsize}}]
    \node (ROOT) [fill=black, circle] {}
      child {node (After) {After} edge from parent node[left] {L\;}}
      child {node (graduation) [fill=black, circle] {}
      {
        child {node {graduation} edge from parent node[left] {P}}
      } edge from parent node[left] {H} }
      child {node {,} edge from parent node[right] {U}}
      child {node (moved) [fill=black, circle] {}
      {
        child {node (John) {John} edge from parent node[left] {A}}
        child {node {moved} edge from parent node[left] {P}}
        child {node [fill=black, circle] {}
        {
          child {node {to} edge from parent node[left] {R}}
          child {node {Copenhagen} edge from parent node[right] {C}}
        } edge from parent node[right] {A} }
      } edge from parent node[right] {H} }
      ;
    \draw[dashed,->,orange,very thick] (graduation) to node [auto] {\scriptsize A} (John);
\end{tikzpicture}}
\end{frame}


\begin{frame}
\frametitle{Data Statistics}
\centering
\def\arraystretch{1.5}
\begin{tabular}{l|r|rrr|r}
    & \multicolumn{1}{c|}{Wiki} & \multicolumn{3}{c|}{20K} & \multicolumn{1}{c}{EWT} \\
    & \multicolumn{1}{c|}{en} & \multicolumn{1}{c}{en} & \multicolumn{1}{c}{fr} & \multicolumn{1}{c|}{de} & \multicolumn{1}{c}{en} \\
    \hline
    \# sentences&5,141&492&492&6,514&3,520 \\
    \# tokens&158,739&12,638&13,021&144,529&51,042 \\
    \hline
    \# {\color{blue} non-terminal nodes}&62,002&4,699&5,110&51,934&18,156 \\
    \% {\color{red}discontinuous}&1.71&3.19&4.64&8.87&3.87 \\
    \% {\color{orange}reentrant}&1.84&0.89&0.65&0.31&0.83 \\
    \hline
    \# edges&208,937&16,803&17,520&187,533&60,739 \\
    \% primary&97.40&96.79&97.02&97.32&97.32 \\
    \% remote&2.60&3.21&2.98&2.68&2.68
\end{tabular}
\end{frame}


\begin{frame}
\frametitle{Evaluation}
\begin{adjustbox}{frame,scale=.75,center}
    \begin{tikzpicture}[level distance=12mm, sibling distance=15mm, ->,
        every circle node/.append style={fill=black},
        edge from parent/.append style={nodes={font=\scriptsize}},
        edge from parent path={(\tikzparentnode.center) -- (\tikzchildnode.north)}]
      \tikzstyle{word} = [font=\rmfamily,color=black]
      \node at (0,.7) {True (human-annotated) graph};
      \node (ROOT) at (0,0) [circle] {}
        child {node (After) [word] {After} edge from parent node[left] {L}}
        child {node (graduation) [circle] {}
        {
          child {node [word] {graduation} edge from parent node[left] {P}}
        } edge from parent node[left] {H} }
        child {node [word] {,} edge from parent node[right] {U}}
        child {node (moved) [circle] {}
        {
          child {node (John) [word] {John} edge from parent node[left] {A}}
          child {node [word] {moved} edge from parent node[left] {P}}
          child {node [circle] {}
          {
            child {node [word] {to} edge from parent node[left] {R}}
            child {node [word] {Copenhagen} edge from parent node[right] {C}}
          } edge from parent node[right] {A} }
        } edge from parent node[right] {H} }
        ;
      \draw[dashed,->] (graduation) to node [auto] {\scriptsize A} (John);
      \node at (8,.7) {Automatically predicted graph for the same text};
      \node (ROOT_) at (7,0) [circle] {}
        child {node (After_) [word] {After} edge from parent node[left] {L}}
        child {node (graduation_) [circle] {}
        {
          child[alt=<2>{red}{}] {node [word] {graduation} edge from parent node[left] {S}}
        } edge from parent node[left] {H} }
        child {node [word] {,} edge from parent node[right] {U}}
        child {node (moved) [circle,xshift=3mm,yshift=-7mm] {}
        {
          child {node (John_) [word] {John} edge from parent node[left] {A}}
          child {node [word] {moved} edge from parent node[left] {P}}
          child[alt=<2>{red}{}] {node [word] {to} edge from parent node[left] {F}}
          child[alt=<2>{red}{}] {node (Copenhagen_) [word] {Copenhagen} edge from parent node[right] {A}}
        } edge from parent node[right] {H} }
        ;
      \draw[dashed,->] (graduation_) to node [auto] {\scriptsize A} (John_);
      \draw[bend left,dashed,->,alt=<2>{red}{}] (graduation_) to[in=90] node [auto] {\scriptsize A} (Copenhagen_);
    \end{tikzpicture}
\end{adjustbox}
\vfill

\begin{enumerate}
  \item Match primary edges between the graphs by terminal yield and label.
  \item Calculate \textbf{precision, recall and F1} scores.
  \item Repeat for remote edges.
\end{enumerate}

\pause
\vfill
\begin{adjustbox}{center}
    \begin{tabular}{c|c|c}
        \multicolumn{3}{l}{Primary} \\
        \textbf{P} & \textbf{R} & \textbf{F1} \\ \hline
        $\frac69=67\%$ & $\frac6{10}=60\%$ & 64\%
    \end{tabular}
    \hspace{1cm}
    \begin{tabular}{c|c|c}
        \multicolumn{3}{l}{Remote} \\
        \textbf{P} & \textbf{R} & \textbf{F1} \\ \hline
        $\frac12=50\%$ & $\frac11=100\%$ & 67\%
    \end{tabular}
\end{adjustbox}
\end{frame}

\end{document}
